%%%%%%%%%%%%%%%%%%%%%%%%%%%%%%%%%%%%%%%%%%%%%%%%%%%%%%%%%%%%%%%%%%
% Artigo segundo as normais mais atualizadas da ABNT
% Adaptado do projeto ABNTeX2 (que nao esta totalmente sincronizada com as normas da ABNT)
% Autor: Berg Dantas (bergdantas@msn.com)
%%%%%%%%%%%%%%%%%%%%%%%%%%%%%%%%%%%%%%%%%%%%%%%%%%%%%%%%%%%%%%%%%%
\documentclass[article,12pt,oneside,a4paper,english,brazil,sumario=tradicional]{abntex2}		
% Pacotes usados
\usepackage{times}%Usa a fonte Latin Modern
\usepackage[T1]{fontenc}%Selecao de codigos de fonte.
\usepackage[utf8]{inputenc}%Codificacao do documento
\usepackage{indentfirst}%Indenta o primeiro parágrafo de cada seção.
\usepackage{nomencl}%Lista de simbolos
\usepackage{color}%Controle das cores
\usepackage{graphicx}%Inclusão de gráficos
\usepackage{microtype}%Para melhorias de justificação
\usepackage{lipsum}%Para geração de dummy text
\usepackage[abnt-emphasize=bf,abnt-and-type=e,alf]{abntex2cite}%Citações ABNT
\usepackage{mathptmx}
%\usepackage[bottom=2cm,top=3cm,left=3cm,right=2cm]{geometry}

% Configuracoes do documento
\graphicspath{{./Figuras/}}%Images na pasta "Figuras"
\setsecheadstyle{\bfseries \normalsize \uppercase}
\setsubsecheadstyle{\normalsize \uppercase}
\setsubsubsecheadstyle{\bfseries \normalsize}
\setlrmarginsandblock{3cm}{2cm}{*}%Margens esquerda-direita
\setulmarginsandblock{3cm}{2cm}{*}%Margens cima-baixo
\checkandfixthelayout
\setlength{\parindent}{1.25cm}%paragrafo
\OnehalfSpacing%espacamento de 1,5
\setlength{\ABNTEXcitacaorecuo}{4cm}%recuo citacao direta +3

\begin{document}
\selectlanguage{brazil} % Seleciona o idioma do documento
\frenchspacing % Retira espaço extra obsoleto entre as frases.

\begin{center}
%TITULO
	\uppercase{\bfseries{Artigo Traceroute}}
	\vspace{12pt}
\end{center}

\begin{flushright}
%AUTOR - Pode-se contar com infinitos autores   :)
    Rafael Gonçalves de Oliveira Viana\footnote{UFMS, rafaelgov95@gmail.com}
	\\
	%Autor B\footnote{, autorA@autor.com}
	\vspace{12pt}
\end{flushright}

\begin{footnotesize}
\SingleSpacing
\noindent
\small{\textbf{Resumo:}}
\noindent
\small
%TEXTO DO RESUMO (em português}
A Internet é uma agregação grande e complexa de hardware de rede, conectada entre eles por gateways. Seguir a rota que os pacotes seguem (ou encontrar o gateway que está descartando seus pacotes) pode ser difícil. 
Nesse artigo estaremos abortando o Trouceroute porém, será levantado protolocos que o trouceroute utiliza para seu funcionamento porém muito breve.

\noindent
%PALAVRAS-CHAVE} 
\textbf{Palavras-chave}: Roteamento, IP, UDP ,TCP SYN, Ping.
\end{footnotesize}

\textual
\pagestyle{simple}
\aliaspagestyle{chapter}{simple}


\section{Introdu\c c\~ao}
\label{secIntroducao}
\normalsize
% TEXTO DA INTRODUCAO
A Internet é uma agregação grande e complexa de hardware de rede, conectada entre eles por gateways. Seguir a rota que os pacotes seguem (ou encontrar o gateway que está descartando seus pacotes) pode ser difícil. 
Nesse artigo estaremos abortando o Trouceroute porém, será levantado protolocos que o trouceroute utiliza para seu funcionamento porém muito breve.

\section{Desenvolvimento}
% TEXTO DO DESENVOLVIMENTO

\subsection{IPv4}
O IP é o elemento comum encontrado na Internet pública dos dias de hoje. É descrito no RFC 791 da IETF, que foi pela primeira vez publicado em Setembro de 1981. Este documento descreve o protocolo da camada de rede mais popular e atualmente em uso. Esta versão do protocolo é designada de versão 4, ou IPv4. O IPv6 tem endereçamento de origem e destino de 128 bits, oferecendo mais endereçamentos que os 32 bits do IPv4.

Ente os 12 campos do IPv4 descrito no RFC 791 falaremos do TTL é campo de oito bits, o TTL (time to live, ou seja, tempo para viver) ajuda a prevenir que os datagramas persistam (ex. andando aos círculos) numa rede. Historicamente, o campo TTL limita a vida de um datagrama em segundos, mas tornou-se num campo de contagem de nós caminhados. Cada comutador de pacotes que um datagrama atravessa decrementa o campo TTL em um valor. Quando o campo TTL chega a zero, o pacote não é seguido por um comutador de pacotes e é descartado.
O propósito do TTL é evitar que datagramas entrem em um loop de roteamento, o que pode ocorrer devido a algum tipo de falha durante o roteamento dos pacotes. Quando um roteador recebe um datagrama cujo TTL é igual a 0 (zero), ele não o encaminhará mais. Em vez disso, o roteador irá descartar o pacote e enviar de volta ao host que o originou uma mensagem ICMP do tipo Tempo Excedido. Essa mensagem contém o endereço IP do roteador como endereço de origem - e esse é o segredo do traceroute.\cite{article2016}

%\vspace{24pt} %dois espacos de 12
\subsection{ICMP}

O ICMP é um protocolo integrante do Protocolo IP, definido pela RFC 792, e utilizado para fornecer relatórios de erros ao host que deu origem aos pacotes enviados na rede. Qualquer computador que utilize o protocolo IP precisa aceitar as mensagens ICMP e alterar o seu comportamento de acordo com o erro relatado. Os gateways (roteadores) devem também estar programados para enviar mensagens ICMP quando receberem pacotes que provoquem algum tipo de erro ou detectarem algum problema listado no protocolo ICMP. O ICMP é transportado no campo de dados do pacote IP e identificado como tipo de protocolo “1” pelo cabeçalho do IP.

As principais mensagens de erro ou informacionais do ICMP geralmente são enviadas automaticamente em uma das seguintes situações:

Um pacote IP não consegue chegar ao seu destino, por exemplo, quando o tempo de vida (TTL) do pacote está expirado (o contador chegou à zero). Esta mensagem é o tempo de vida expirado ou “time exceeded”.

O roteador não consegue retransmitir os pacotes na frequência adequada, ou seja, o roteador está congestionado (mensagem “source quench”).

O roteador indica uma rota melhor para o host que está enviando pacotes (mensagem de redirecionamento de rota ou “redirect”).

Quando um host de destino ou rota não está alcançável (mensagem “destination unreachable” ou destino inalcançável).

Quando o host ou o roteador descobrem um erro de sintaxe no cabeçalho do IP (mensagem “parameter problem”). 

Existem diversas outras mensagens que o ICMP pode fornecer e cada uma é representada por um tipo ou código.

 
 \section{Traceroute}
 O utilitário traceroute, que foi escrito por Van Jacobson em 1987, é uma ferramenta de diagnóstico que nos permite ver a rota que datagramas IP seguem quando são enviados de um host a outro. O traceroute faz uso do protocolo ICMP e do campo TTL no cabeçalho IP do datagrama. O valor a ser usado neste campo varia entre os sistemas operacionais, sendo comuns os valores 128 para sistemas Windows e 64 para sistemas baseados em Unix, como o Linux (em pacotes normais; o traceroute utiliza valores totalmente diferentes).
 
 \subsection{Funcionamento }
 Traceroute  utiliza o campo "time to live" do protocolo IP e tenta obter uma resposta ICMP TIME\_EXCEEDED de cada gateway ao longo do caminho para algum host.


\section{Considera\c c\~oes finais}
% TEXTO DA CONCLUSAO

Considera\c c\~oes finais considera\c c\~oes finais considera\c c\~oes finais considera\c c\~oes finais considera\c c\~oes finais considera\c c\~oes finais.

\renewcommand{\bibsection}{\section*{REFER\^ENCIAS BIBLIOGR\'AFICAS}}
\bibliographystyle{abntex2-alf}
\bibliography{Referencias}
\end{document}