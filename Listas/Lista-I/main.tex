% abtex2-modelo-artigo.tex, v-1.9.2 laurocesar
% Copyright 2012-2014 by abnTeX2 group at http://abntex2.googlecode.com/ 
%

% ------------------------------------------------------------------------
% ------------------------------------------------------------------------
% abnTeX2: Modelo de Artigo Acadêmico em conformidade com
% ABNT NBR 6022:2003: Informação e documentação - Artigo em publicação 
% periódica científica impressa - Apresentação
% ------------------------------------------------------------------------
% ------------------------------------------------------------------------

\documentclass[
	% -- opções da classe memoir --
	article,			% indica que é um artigo acadêmico
	11pt,				% tamanho da fonte
	oneside,			% para impressão apenas no verso. Oposto a twoside
	a4paper,			% tamanho do papel. 
	% -- opções da classe abntex2 --
	%chapter=TITLE,		% títulos de capítulos convertidos em letras maiúsculas
	%section=TITLE,		% títulos de seções convertidos em letras maiúsculas
	%subsection=TITLE,	% títulos de subseções convertidos em letras maiúsculas
	%subsubsection=TITLE % títulos de subsubseções convertidos em letras maiúsculas
	% -- opções do pacote babel --
	english,			% idioma adicional para hifenização
	brazil,				% o último idioma é o principal do documento
	sumario=tradicional
	]{abntex2}


% ---
% PACOTES
% ---

% ---
% Pacotes fundamentais 
% ---
\usepackage{lmodern}			% Usa a fonte Latin Modern
\usepackage[T1]{fontenc}		% Selecao de codigos de fonte.
\usepackage[utf8]{inputenc}		% Codificacao do documento (conversão automática dos acentos)
\usepackage{indentfirst}		% Indenta o primeiro parágrafo de cada seção.
\usepackage{nomencl} 			% Lista de simbolos
\usepackage{color}				% Controle das cores
\usepackage{graphicx}			% Inclusão de gráficos
\usepackage{microtype} 			% para melhorias de justificação
% ---
		
% ---
% Pacotes adicionais, usados apenas no âmbito do Modelo Canônico do abnteX2
% ---
\usepackage{lipsum}				% para geração de dummy text
% ---
		
% ---
% Pacotes de citações
% ---
\usepackage[brazilian,hyperpageref]{backref}	 % Paginas com as citações na bibl
\usepackage[alf]{abntex2cite}	% Citações padrão ABNT
% ---

% ---
% Configurações do pacote backref
% Usado sem a opção hyperpageref de backref
\renewcommand{\backrefpagesname}{Citado na(s) página(s):~}
% Texto padrão antes do número das páginas
\renewcommand{\backref}{}
% Define os textos da citação
\renewcommand*{\backrefalt}[4]{
	\ifcase #1 %
		Nenhuma citação no texto.%
	\or
		Citado na página #2.%
	\else
		Citado #1 vezes nas páginas #2.%
	\fi}%
% ---

% ---
% Informações de dados para CAPA e FOLHA DE ROSTO
% ---
\titulo{Artigo Traceroute }
\autor{ Rafael Gonçalves de Oliveira Viana}
\local{Brasil}
\data{2017}
% ---

% ---
% Configurações de aparência do PDF final

% alterando o aspecto da cor azul
\definecolor{blue}{RGB}{41,5,195}

% informações do PDF
\makeatletter
\hypersetup{
     	%pagebackref=true,
		pdftitle={\@title}, 
		pdfauthor={\@author},
    	pdfsubject={Modelo de artigo científico com abnTeX2},
	    pdfcreator={LaTeX with abnTeX2},
		pdfkeywords={abnt}{latex}{abntex}{abntex2}{atigo científico}, 
		colorlinks=true,       		% false: boxed links; true: colored links
    	linkcolor=blue,          	% color of internal links
    	citecolor=blue,        		% color of links to bibliography
    	filecolor=magenta,      		% color of file links
		urlcolor=blue,
		bookmarksdepth=4
}
\makeatother
% --- 

% ---
% compila o indice
% ---
\makeindex
% ---

% ---
% Altera as margens padrões
% ---
\setlrmarginsandblock{3cm}{3cm}{*}
\setulmarginsandblock{3cm}{3cm}{*}
\checkandfixthelayout
% ---

% --- 
% Espaçamentos entre linhas e parágrafos 
% --- 

% O tamanho do parágrafo é dado por:
\setlength{\parindent}{1.3cm}

% Controle do espaçamento entre um parágrafo e outro:
\setlength{\parskip}{0.2cm}  % tente também \onelineskip

% Espaçamento simples
\SingleSpacing

% ----
% Início do documento
% ----
\begin{document}

% Retira espaço extra obsoleto entre as frases.
\frenchspacing 

% ----------------------------------------------------------
% ELEMENTOS PRÉ-TEXTUAIS
% ----------------------------------------------------------

%---
%
% Se desejar escrever o artigo em duas colunas, descomente a linha abaixo
% e a linha com o texto ``FIM DE ARTIGO EM DUAS COLUNAS''.
% \twocolumn[    		% INICIO DE ARTIGO EM DUAS COLUNAS
%
%---
% página de titulo
\maketitle

% resumo em português
\begin{resumoumacoluna}
 Este artigo tem como objetivo uma apresentação do funcionamento do Traceroute uma ferramenta de gerenciamento de rede. O traceroute é utilizado para detectar falhas como, por exemplo, gateways intermediários que descartam pacotes ou rotas que excedem a capacidade de um datagrama ip entre outras. Também será de vital importância para esse artigo uma abortagem ao seus protrocolo e o funcionamento dos mesmos.
 
 
 \vspace{\onelineskip}
 
 \noindent
 \textbf{Palavras-chaves}: Roteamento, IP, UDP ,TCP SYN, Ping.
\end{resumoumacoluna}

% ]  				% FIM DE ARTIGO EM DUAS COLUNAS
% ---

% ----------------------------------------------------------
% ELEMENTOS TEXTUAIS
% ----------------------------------------------------------
\textual

% ----------------------------------------------------------
% Introdução
% ----------------------------------------------------------
\section*{Introdução}
\addcontentsline{toc}{section}{Introdução}
A Internet é uma agregação grande e complexa de hardware de rede, conectada entre eles por gateways. Seguir a rota que os pacotes seguem (ou encontrar o gateway que está descartando seus pacotes) pode ser difícil. 
Nesse artigo estaremos abortando o Trouceroute porém, será levantado protolocos que o trouceroute utiliza para seu funcionamento porém muito breve.

\section{Protocolos}

\subsection{IPv4}
O IP é o elemento comum encontrado na Internet pública dos dias de hoje. É descrito no RFC 791 da IETF, que foi pela primeira vez publicado em Setembro de 1981. Este documento descreve o protocolo da camada de rede mais popular e atualmente em uso. Esta versão do protocolo é designada de versão 4, ou IPv4. O IPv6 tem endereçamento de origem e destino de 128 bits, oferecendo mais endereçamentos que os 32 bits do IPv4.

Ente os 12 campos do IPv4 descrito no RFC 791 falaremos do TTL é campo de oito bits, o TTL (time to live, ou seja, tempo para viver) ajuda a prevenir que os datagramas persistam (ex. andando aos círculos) numa rede. Historicamente, o campo TTL limita a vida de um datagrama em segundos, mas tornou-se num campo de contagem de nós caminhados. Cada comutador de pacotes que um datagrama atravessa decrementa o campo TTL em um valor. Quando o campo TTL chega a zero, o pacote não é seguido por um comutador de pacotes e é descartado.

\subsection{ICMP}

O ICMP é um protocolo integrante do Protocolo IP, definido pela RFC 792, e utilizado para fornecer relatórios de erros ao host que deu origem aos pacotes enviados na rede. Qualquer computador que utilize o protocolo IP precisa aceitar as mensagens ICMP e alterar o seu comportamento de acordo com o erro relatado. Os gateways (roteadores) devem também estar programados para enviar mensagens ICMP quando receberem pacotes que provoquem algum tipo de erro ou detectarem algum problema listado no protocolo ICMP. O ICMP é transportado no campo de dados do pacote IP e identificado como tipo de protocolo “1” pelo cabeçalho do IP.

As principais mensagens de erro ou informacionais do ICMP geralmente são enviadas automaticamente em uma das seguintes situações:

Um pacote IP não consegue chegar ao seu destino, por exemplo, quando o tempo de vida (TTL) do pacote está expirado (o contador chegou à zero). Esta mensagem é o tempo de vida expirado ou “time exceeded”.

O roteador não consegue retransmitir os pacotes na frequência adequada, ou seja, o roteador está congestionado (mensagem “source quench”).

O roteador indica uma rota melhor para o host que está enviando pacotes (mensagem de redirecionamento de rota ou “redirect”).

Quando um host de destino ou rota não está alcançável (mensagem “destination unreachable” ou destino inalcançável).

Quando o host ou o roteador descobrem um erro de sintaxe no cabeçalho do IP (mensagem “parameter problem”). 

Existem diversas outras mensagens que o ICMP pode fornecer e cada uma é representada por um tipo ou código.\cite{dltec}

\subsection{Recuo do ambiente }



\section{Traceroute}


Traceroute  utiliza o campo "time to live" do protocolo IP e tenta obter uma resposta ICMP TIME\_EXCEEDED de cada gateway ao longo do caminho para algum host.



\section{Mais exemplos no Modelo Canônico de Trabalhos Acadêmicos}

Este modelo de artigo é limitado em número de exemplos de comandos, pois são
apresentados exclusivamente comandos diretamente relacionados com a produção de
artigos.


\section{Consulte o manual da classe \textsf{abntex2}}


% ---
% Finaliza a parte no bookmark do PDF, para que se inicie o bookmark na raiz
% ---
\bookmarksetup{startatroot}% 
% ---

% ---
% Conclusão
% ---
\section*{Considerações finais}
\addcontentsline{toc}{section}{Considerações finais}

\lipsum[1]

\begin{citacao}
\lipsum[2]
\end{citacao}

\lipsum[3]

\bibliography{abntex2-modelo-references}


\end{document}
