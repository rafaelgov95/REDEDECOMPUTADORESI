\documentclass[12pt]{article}

\usepackage{sbc-template}

\usepackage{graphicx,url}
\usepackage{float}
\usepackage[brazil]{babel}   
%\usepackage[latin1]{inputenc}  
\usepackage[utf8]{inputenc}  
% UTF-8 encoding is recommended by ShareLaTex
\usepackage{xcolor}
% Definindo novas cores
\definecolor{verde}{rgb}{0.25,0.5,0.35}
\definecolor{jpurple}{rgb}{0.5,0,0.35}
% Configurando layout para mostrar codigos Java
\usepackage{listings}
\lstset{
	language=Java,
	basicstyle=\ttfamily\small, 
	keywordstyle=\color{jpurple}\bfseries,
	stringstyle=\color{red},
	commentstyle=\color{verde},
	morecomment=[s][\color{blue}]{/**}{*/},
	extendedchars=true, 
	showspaces=false, 
	showstringspaces=false, 
	numbers=left,
	numberstyle=\tiny,
	breaklines=true, 
	backgroundcolor=\color{cyan!10}, 
	breakautoindent=true, 
	captionpos=b,
	xleftmargin=0pt,
	tabsize=4
}
\pagestyle{empty}
     
\sloppy
\title{Sistema Cliente Com Interface Gráfica Para Conexão FTP \\ Exercício Computacional I - Redes De Computadores}

\author{Rafael Gonçalves de Oliveria Vianal\inst{1} }


\address{Sistemas de Informação -- Universidade Federal do Mato Grosso do Sul
	(UFMS)\\
  	Caixa Postal 79400-000 -- Coxim -- MS -- Brazil
  \email{rafael.viana@aluno.ufms.br}
}

\begin{document} 

\maketitle

     
\begin{resumo} 	
  Este relatório descreve como foi implementado um sistema cliente com interface gráfica para conexão FTP, utilizando JavaFX em conjunto do Apache Commons Net 3.6, uma biblioteca de conexão FTP.
\end{resumo}


\section{JavaFX}
Foi escolhido o JavaFX para criar uma interface gráfica onde o usuário terá um melhor desempenho, ao utilizar o sistema.
Para criar um sistema elegante foi utilizado uma biblioteca com novos elementos CSS, a bibliteca utilizada para essa finalidade foi  a \textbf{JFoenix}, essa bibliteca é open source. Para icones foi utilizado a biblioteca fontawesomefx-8.9 essa biblioteca é open source.
	
\section{Apache Commons Net 3.6} \label{sec:firstpage}

Para melhor desempenho nas conexões ftp, foi utilizada a biblioteca de conexão FTP da Apache Commons, onde a mesma se encontra atualmente na versão 3.6.

\section{Problematica}
O trabalho proposto tem como objetivo criar um client FTP, no qual tenha como Adicionar, Renomear, realizar Download e Upload de arquivos e pastas. Tendo como restrição o sistema deve apenas deixar criar 5 pastas e 2 arquivos por diretório, sendo que no máximo deve ser criados 3 níveis de diretórios.
 
A implementação de restrições no software cliente e não no software servidor, coloca a aplicação em risco, um usuário mal intencionado poderia utilizar outro software cliente para acessar os serviços do software servidor, já que o mesmo não possui restrições, assim o usuário mal intencionado passaria a ter privilégios.

Assim podemos observar uma falha na segurança dessas restrições, porém como essa aplicação e para fins acadêmicos, não focaremos nessa questão.


\section{Metodologia}
Como a aplicação é pequena a estrutura da mesma, foi dividida em 3 pacotes sendo eles: Cliente, Icons e Socket, mostradas na Figura \ref{fig:01}

\begin{enumerate}
	
\item{A pasta cliente é responsável pela parte de Interface Gráfica do Usuário, envolvendo controles e Views.}
\item{A pasta Icons armazena icons utilizados na pasta cliente.}
\item{A pasta socket e responsável por toda comunicação da Lib da Apache com os controles do cliente.}

\end{enumerate}

\begin{figure}[H]
	\centering
	\includegraphics[width=.3\textwidth]{Imagens/001.png}
	\caption{ Imagem da Tela de Login.}
	\label{fig:01}
\end{figure}


Primeiramente a Class Main e invocada, chamando a Scene do Login.fxml, o controller do login e responsável por fornecer o necessario para o FTPClient poder fazer a conexão.

\begin{lstlisting}

public class Main  extends Application{

@Override
	public void start(Stage stage) throws Exception {
	Parent root = FXMLLoader.load(getClasssingleton().getResource("/cliente/Login.fxml"));
	Scene scene = new Scene(root);
	stage.setScene(scene);
	stage.show();
}


public static void main(String[] args) {
	launch(args);

}

}

\end{lstlisting}

\begin{figure}[H]
	\centering
	\includegraphics[width=.9\textwidth]{Imagens/01.png}
	\caption{ Imagem da Tela de Login.}
	\label{fig:02}
\end{figure}

Com a tela de login aberta o usuário entra com as informações login , senha, endereço do host, port do host mostrado no código abaixo e na Figura \ref{fig:02} .
\vspace{.4cm}
\begin{lstlisting}
private void login(ActionEvent event) {

	btnLogin.setVisible(false);
	imgProgress.setVisible(true);
	
	PauseTransition pauseTransition = new PauseTransition();
	pauseTransition.setDuration(Duration.seconds(3));
	pauseTransition.setOnFinished(ev -> {

try {
	
		int reply = FTPFactory.getInstance().FTPConecta(txtHostName.getText(),Integer.parseInt(txtHostPort.getText()),this.txtUsername.getText(),this.txtPassword.getText());
		System.out.println("Igual:" + reply);
		
		if (reply == 230) {
		
			btnLogin.getScene().getWindow().hide();
			completeLogin();
		
		} else {
		
			imgProgress.setVisible(false);
			btnLogin.setVisible(true);
			JOptionPane.showMessageDialog(null,"Erro Senha ou Usuário incorreto !!", "Erro ao Logar",JOptionPane.ERROR_MESSAGE);
		}

	} catch (IOException ex) {
		Logger.getLogger(LoginController.class.getName()).log(Level.SEVERE, null, ex);
	} catch (Exception ex) {
		Logger.getLogger(LoginController.class.getName()).log(Level.SEVERE, null, ex);
	}

});
  pauseTransition.play();
}

private void completeLogin() throws IOException {
	
	imgProgress.setVisible(false);
	Stage dashboardStage = new Stage();
	dashboardStage.setTitle("");
	Parent root = FXMLLoader.load(getClass().getResource("Navegador.fxml"));
	Scene scene = new Scene(root);
	dashboardStage.setScene(scene);
	dashboardStage.show();
	
}

\end{lstlisting}


Para poder realizar a comunicação entre as classes e o FTPClient da apache, foi criado uma classe Singleton chamada de FTPFactory onde a mesma cria uma getInstance de FTPClient, podendo ser chamada de qualquer classe sem ter que ser instanciada novamente, o que irá ocasionar a perda da conexão FTP.

\begin{lstlisting}

public class FTPFactory {
	
	private final FTPClient ftp;
	private TreeItem<FTPFile> file;
		
	private FTPFactory() {
			this.ftp = new FTPClient();
	}
	
	public static FTPFactory getInstance() {
	
		return FTPFactoryHolder.INSTANCE;
	}


	/**
	* Classe privada que armazena a única instância de FTPFactory.
	*/
	
	private static class FTPFactoryHolder {
	
		private static final FTPFactory INSTANCE = new FTPFactory();
	}


	public FTPClient getFTP() {
		return this.ftp;
	}


	public boolean Excluir(FTPFile file) {
	try {
			if (file.isDirectory()) {
				System.out.println(file.getLink());
				return ftp.removeDirectory(file.getLink());
			} else {
				System.out.println(file.getLink());
				return ftp.deleteFile(file.getLink());
			}
	} catch (IOException e) {
		e.printStackTrace();
		}	
		return false;
	}


	public int FTPConecta(String host, int port, String user, String pwd) throws Exception {
		int reply;
		ftp.connect(host, port);
		reply = ftp.getReplyCode();
		if (!FTPReply.isPositiveCompletion(reply)) {
			ftp.disconnect();
			throw new Exception("Exception in connecting to FTP Server");
		}
		ftp.login(user, pwd);
		reply = ftp.getReplyCode();
		ftp.setFileType(FTPClient.BINARY_FILE_TYPE);
		ftp.enterLocalPassiveMode();
		ftp.setAutodetectUTF8(true);
		return reply;
	}


	public void disconnect() {
		if (this.ftp.isConnected()) {
		try {
			this.ftp.logout();
			this.ftp.disconnect();
			} catch (IOException f) {
	
			}
	    }
	}
}	

\end{lstlisting}

Após a Class login em conjunto com a Class FTPFactor, validar os dados de usuário a tela de navegação e aberta, essa tela nada mais é do que um conjunto de botões em uma Grid a esquerda e um TreeView do JavaFX no centro para poder navegar pela estutura dos diretórios.

\begin{figure}[H]
	\centering
	\includegraphics[width=0.8\textwidth]{Imagens/02.png}
	\caption{ Imagem da Tela de Navegação.}
	\label{fig:03}
\end{figure}


Toda lógica de abastecimento da TreeView e a criação dos TreeItem utilizados na TreeView estão na Class  navegação, onde inicia a criação dos TreeItem da toda a estrutura da arvore como ADICIONAR, REMOVER, LISTAR, EDITAR, DOWNLOAD e UPLOAD de arquivos/pastas, o método CarregarFiles(), cria recursivamente Nodes/TreeItem apartir dos files no servidor FTP, posteriormente encadeando os TreeItems e lançando-os na TreeView, facilitando assim a navegação.

\begin{lstlisting}

private void Navegacao() throws IOException {
	FTPFile files[];
	TreeItem<FTPFile> treeRoot;
	files = FTPFactory.getInstance().getFTP().listFiles();
	Tree.setEditable(true);
	if (files != null && files.length > 0) {
		files[0].setRawListing(FTPFactory.getInstance().getFTP().getPassiveHost());
		treeRoot = CarregarFiles(files[0], true);
	} else {
		FTPFile file = new FTPFile();
		file.setType(FTPFile.DIRECTORY_TYPE);
		file.setLink(FTPFactory.getInstance().getFTP().printWorkingDirectory());
		file.setRawListing(FTPFactory.getInstance().getFTP().getPassiveHost());
		treeRoot = new TreeItem<>(file, new ImageView(computador));
	}
	
	Tree.getSelectionModel().select(treeRoot);	
	Tree.setRoot(treeRoot);
	
	//Os Botões ADD,RENAME,DELETE,UPLOAD e DOWNLOAD estão nessa classe, foi retirada para um melhor o entendimento.

}


public TreeItem<FTPFile> CarregarFiles(FTPFile directory, boolean v) throws IOException {
	
	TreeItem<FTPFile> root;

	if (v) {
		directory.setType(FTPFile.DIRECTORY_TYPE);
		directory.setLink(FTPFactory.getInstance().getFTP().printWorkingDirectory());
		root = new TreeItem<FTPFile>(directory, new ImageView(computador));
    } else {
		root = new TreeItem<FTPFile>(directory, new ImageView(pasta));
	}
	root.setExpanded(true);
	FTPFile[] files = FTPFactory.getInstance().getFTP().listFiles();
	for (FTPFile f : files) {
		System.out.println("Carregando .. " + f.getName());
	    if (f.isDirectory()) {
			FTPFactory.getInstance().getFTP().changeWorkingDirectory(f.getName());
			f.setLink(FTPFactory.getInstance().getFTP().printWorkingDirectory());
			root.getChildren().add(CarregarFiles(f, false));
		} else {
		f.setLink(FTPFactory.getInstance().getFTP().printWorkingDirectory() + separador + f.getName());
		root.getChildren().add(new TreeItem<FTPFile>(f, new ImageView(this.arquivo)));
		}
	}
	FTPFactory.getInstance().getFTP().changeToParentDirectory(); 
	return root;
}

\end{lstlisting}

Um dos objetivos do trabalho era limitar o cliente FTP, fazendo com que usuários que utilizar o sistema só poderam criar 5 pastas e 2 arquivos por diretório, sendo que poderá criar no máximo 3 níveis de diretórios.

A classe Limitador do pacote socket e responsável por fazer a armazenagem da quantidade máxima de pastas e arquivos. Sendo utilizada na classe navegação do pacote cliente, em conjunto dos métodos limiteArquivo(), limitePasta()  e limiteNivel().
 
 \begin{lstlisting}
public class Limitador {

	private int p;
	private int a;
	
	public Limitador(int p, int a) {
		this.p = p;
		this.a = a;
	}
	
	public int getMP() {
		return p;
	}
	
	public int getMA() {
		return a;
	}

}

// Os métodos abaixo pertencem Class NavegadorControoler

private boolean limiteArquivo() throws IOException {
	int num = FTPFactory.getInstance().getFTP().listDirectories().length;
	int numa = FTPFactory.getInstance().getFTP().listFiles().length - num;
	return numa < limite.getMA();
}

private boolean limitePasta() throws IOException {
	int num_diretorios = FTPFactory.getInstance().getFTP().listDirectories().length;
	return num_diretorios < limite.getMP();
}

private int limiteNivel() throws IOException {
	int num = FTPFactory.getInstance().getFTP().printWorkingDirectory().split("/").length;
	return num;
}


\end{lstlisting}
	Para deletar uma pasta deve-se utilizar o método removeDirectory(Caminho), ou para deletar um arquivo utiliza-se o método deleteFile(Caminho) do FTPClient Apache. Porém para deletar uma pasta que contém N sub pastas deve-se percorrer recursivamente as pastas, até a folha mais baixa e vim apagando debaixo para cima cada pasta ou arquivo de cada pasta recursivamente pelo método DeletarRecursivo(TreeItem).
	Para renomear um arquivo ou pasta utiliza-se o método rename(caminho,novonome) da Apache. Porém para mudar os links das sub pastas de um diretório  
\begin{lstlisting}

public boolean DeletarRecursivo(TreeItem<FTPFile> a) {
	boolean flag = false;
	
	if (a.getChildren().isEmpty()) {
		if (FTPFactory.getInstance().Excluir(a.getValue())) {
			return true;
		}
	}else {
	
		for (TreeItem<FTPFile> iterator: a.getChildren()){
			DeletarRecursivo(iterator);
		}
		
		if (FTPFactory.getInstance().Excluir(a.getValue())) {
			return true;
		}
	}
	return false;
}

public void RenameRecursivao(TreeItem<FTPFile> a) {
	String novolink;
	for (Iterator<TreeItem<FTPFile>> iterator = a.getChildren().iterator(); iterator.hasNext(); ) {
		TreeItem<FTPFile> c = iterator.next();
		novolink = a.getValue().getLink() + separador + c.getValue().getName();
		c.getValue().setLink(novolink);
		if (!c.getChildren().isEmpty()) {
			RenameRecursivao(c);
		}
	}
}
\end{lstlisting}


\end{document}


